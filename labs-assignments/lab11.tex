\documentclass[12pt]{article}
\usepackage{amsmath}
\usepackage{listings}
\title{Computer Graphics: Lab class 11 - Motion and animation exercises}
\author{hlysig}
\begin{document}
\maketitle

\section{Calculating position of object moved with a Bezier curve}
In this first exercise we solve a simple Bezier curve question. The question is as follows.
We are given the following points.

\begin{eqnarray*}
p_1 &=& (1,4,8)\\
p_2 &=& (4,5,7)\\
p_3 &=& (6,-1, 3)\\
p_4 &=& (12,20,50)
\end{eqnarray*}

Let us assume that a given object is moved using a Bezier curve. The points, $p_1$, $p_2$, $p_3$ and
$p_4$ are the control points for the curve where $p_1$ is the starting point the $p_4$ is the ending point. Let us assume that this animation has the initial time at $20$ and the animation ends at time $50$. Please provide answer and calculation for the following question. What is the position of the object at time $31$?

\subsection{Solution}
In this problem we have four control points and from the lecture on Bézier curves and animation we learned that using Bernstein polynomials of degree $n$ gives us control point factors for a Bézier curve with $n+1$ control points.
\begin{equation}\label{eq:bernstein}
    B^n = \left( \left( 1-t \right)+t \right)^n
\end{equation}
We define Bernstein polynomial as (\ref{eq:bernstein}). In this case we solve (\ref{eq:bernstein}) in degree $3$ where we have four control points.

\begin{equation}\label{eq:thirdbernstein}
    B^3 = (1-t)^3 p_1 + 3(1-t)^2 t p_2 + 3(1-t) t^2 p_3 + t^3 p_4
\end{equation}
That is for control points $p_1$, $p_2$, $p_3$ and $p_4$.

From the problem description we know that the start time is $20$ and the end time is $50$ and
we are asked to find the objects value at time $31$. Let us calculate the $t$ ratio
that we then use with (\ref{eq:thirdbernstein}) to find the position at time $t_{31}$. 

\begin{eqnarray*}
    t &=& \frac{\textnormal{current time} - \textnormal{start time}}{\textnormal{end time} - \textnormal{start time}}\\
    t_{31} &=& \frac{31 - 20}{50-20}\\
    t_{31} &=& \frac{11}{30}
\end{eqnarray*}

With the $t_{31}$ ratio we can can calculate the position $p$ of the box using \ref{eq:thirdbernstein} and our points. Let us call that function $B^3(t)$.

\begin{eqnarray*}
    B^3\left(\frac{11}{30}\right) &=& \left(1-\frac{11}{30}\right)^3 p_1 + 3\left(1-\frac{11}{30}\right)^2 \frac{11}{30} p_2 + 3\left(1-\frac{11}{30}\right) \left(\frac{11}{30}\right)^2 p_3 + \left(\frac{11}{30}\right)^3 p_4\\
    B^3\left(\frac{11}{30}\right) &=& \left(\frac{6859}{27000}\right) p_1 + 3\left(\frac{361}{900}\right) \frac{11}{30} p_2 + 3\left(\frac{19}{30}\right) \frac{121}{900} p_3 + \frac{1331}{27000} p_4\\
    B^3 \left( \frac{11}{30} \right) &=& \frac{6859}{27000} p_1 + 3 \frac{3971}{27000} p_2 + 3 \frac{2299}{27000} p_3 + \frac{1331}{27000} p_4\\
    B^3 \left( \frac{11}{30} \right) &=& \frac{6859}{27000} p_1 + \frac{11913}{27000} p_2 + \frac{6897}{27000} p_3 + \frac{1331}{27000} p_4\\
    B^3 \left( \frac{11}{30} \right) &=& \frac{6859}{27000} (1,4,8) + \frac{11913}{27000} (4,5,7) + \frac{6897}{27000} (6,-1,3) + \frac{1331}{27000} (12,20,50)\\
    B^3\left(\frac{11}{30}\right) &=& (0.25, 1.02, 2.03) + (1.76, 2.21, 3.09) + (1.53, -0.25, 0,767) + (0.60, 0.98, 2.46)\\
    B^3\left(\frac{11}{30}\right) &=& (4.14, 3.96, 8.347)
\end{eqnarray*}

With this control points and with the given start time and end time our object
at time 31 will be at the point $(4.14,3.96,8.347)$.

\section{Motion as function of time}
To do an animation in (real time) 3D we have to make sure it looks the same every
time the program is run. This is obvious with pre-rendered animations but in a real-
time environment we don't know if the same amount of frames will be rendered
each time. Thus we need to keep track of the time that has elapsed since the
program started and let each object update itself accordingly. The will update their
position and orientation (and maybe color and size) not only relative to the last
frame but calculate it exactly as it should be at this point in the program. The change
therefor cannot be an offset but has to be a function of the whole time passed, to
make sure the animation will display the same frame at a given time, regardless of
how many or which frames have already been displayed.
The update function can be as follows.
\begin{lstlisting}[frame=single, language=Java, basicstyle=\footnotesize\ttfamily]
float elapsedTime = 0;
void update() {
  //here's the time from the start of the program
  elapsedTime += Gdx.graphics.getDeltaTime();
  someObject.update(elapsedTime);
}
\end{lstlisting}
elapsedTime is now the time since update was called for the first time. We can also
still keep track of deltaTime between frames if some objects are moved by input.

\subsection{Motion Exercise}
Make a class that draws some object with a position. You can use the box class. This object's draw function
must perform it's own glTranslate( to a position it holds in a class variable
(glTranslate should not be done outside the class, in the core display routine).
Add an update function that takes a variable float elapsedTime which is the time in
seconds. Make this function update the position as some function of time.
You can try
\begin{lstlisting}[frame=single, language=Java, basicstyle=\footnotesize\ttfamily]
Position.y = sin(elapsedTime * pi);
\end{lstlisting}
or,
\begin{lstlisting}[frame=single, language=Java, basicstyle=\footnotesize\ttfamily]
Position.x = elapsedTime * speedConstant;
\end{lstlisting}
Add to the class variables for the rotation around different axis and try to update
those similarly (and add the relevant transformation functions to the display
function).

Add startMotionTime and endMotionTime to the class. Also add Points
startPostition and endPosition. Make the object stay at startPosition until
elapsedTime is more than startMotionTime, then move at an even speed to
endPosition, stopping there at endMotionTime. Between start and end, find

\begin{lstlisting}[frame=single, language=Java, basicstyle=\footnotesize\ttfamily]
t = (elapsedTime - startMotionTime) 
     / (endMotionTime - startMotionTime)
\end{lstlisting}
and use t as a parameter in a linear interpolation between the points.

\begin{equation}
lerp(p_1, p_2, t) = (1-t) \times p_1 + t \times p_2
\end{equation}

Try something similar for the position of the camera, LookAt point of camera,
positions of lights (and colors and more). Go crazy!

\subsection{Bezier curve exercise}
Now you've got a class that moves an object along a path between two points. It
does this by mapping the time between startTime and endTime to a parameter
between 0 and 1. Now use this parameter in a function for a Bezier curve with 4
control points. Try to have the control points within the viewing volume at first to
see the object the whole time. Then experiment with the control points.
\end{document}
